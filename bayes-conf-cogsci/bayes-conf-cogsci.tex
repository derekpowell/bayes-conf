% Template for Cogsci submission with R Markdown

% Stuff changed from original Markdown PLOS Template
\documentclass[10pt, letterpaper]{article}

\usepackage{cogsci}
\usepackage{pslatex}
\usepackage{float}
\usepackage{caption}

% amsmath package, useful for mathematical formulas
\usepackage{amsmath}

% amssymb package, useful for mathematical symbols
\usepackage{amssymb}

% hyperref package, useful for hyperlinks
\usepackage{hyperref}

% graphicx package, useful for including eps and pdf graphics
% include graphics with the command \includegraphics
\usepackage{graphicx}

% Sweave(-like)
\usepackage{fancyvrb}
\DefineVerbatimEnvironment{Sinput}{Verbatim}{fontshape=sl}
\DefineVerbatimEnvironment{Soutput}{Verbatim}{}
\DefineVerbatimEnvironment{Scode}{Verbatim}{fontshape=sl}
\newenvironment{Schunk}{}{}
\DefineVerbatimEnvironment{Code}{Verbatim}{}
\DefineVerbatimEnvironment{CodeInput}{Verbatim}{fontshape=sl}
\DefineVerbatimEnvironment{CodeOutput}{Verbatim}{}
\newenvironment{CodeChunk}{}{}

% cite package, to clean up citations in the main text. Do not remove.
\usepackage{apacite}

% KM added 1/4/18 to allow control of blind submission
\cogscifinalcopy

\usepackage{color}

% Use doublespacing - comment out for single spacing
%\usepackage{setspace}
%\doublespacing


% % Text layout
% \topmargin 0.0cm
% \oddsidemargin 0.5cm
% \evensidemargin 0.5cm
% \textwidth 16cm
% \textheight 21cm

\title{Bayesian confirmation and commonsense notions of evidential
strength}

\usepackage{float}

\author{{\large \bf Derek Powell (dmpowell@asu.edu)} \\ School of Social and Behavioral Sciences \\ 4701 W. Thunderbird Rd. Glendale, AZ 85306 USA \AND {\large \bf Shyam Nair (gsnair@asu.edu)} \\ School of Historical, Philosophical, and Religious Studies \\ 975 S. Myrtle Ave \\ P.O. Box 874302 \\ Tempe, AZ 85287 USA}

\newlength{\cslhangindent}
\setlength{\cslhangindent}{1.5em}
\newenvironment{CSLReferences}%
  {}%
  {\par}

\begin{document}

\maketitle

\begin{abstract}
How can we quantify the degree to which a piece of evidence affects a
person's belief? Philosophers investigating theories of \emph{Bayesian
Confirmation} have identified a plurality of potential measures, each
with their own virtues and shortcomings. Psychologists meanwhile have
largely neglected this question, which has limited their ability to
understand \emph{differential belief updating}, cases where certain
individuals or groups respond to the same evidence in different ways. In
this study, we examine how competing Bayesian confirmation measures
track commonsense notions of evidential strength. We demonstrate how
these measures can be computed from participants' belief reports, and
identify cases where the measures come apart in their characterization
of participants' belief updating. In so doing, this project seeks to
build connections between investigations of psychological belief
updating processes and formal epistemic theories of confirmation.

\textbf{Keywords:}
Belief updating; Bayesian confirmation
\end{abstract}

\hypertarget{introduction}{%
\section{Introduction}\label{introduction}}

Suppose across some shared moment, a child pushes a cup from a table and
watches it fall to the floor, a physics student watches a
counterintuitive demonstration, a scientist compares the fit of two
statistical models, and a philosopher weighs the force of an argument.
When each observes their evidence, how much do those observations change
their beliefs? How can this change be quantified, in principle and in
practice?

On the principled front, epistemologists investigated methods for
quantifying how evidence impacts beliefs by considering different
theories of \emph{Bayesian Confirmation}. These theorists have sought to
define measures of the degree to which some evidence \(E\) confirms a
hypothesis \(H\) relative to background knowledge \(K\). Within this
literature there is no strong consensus as to a particular ``best''
measure, as several competing measures are recognized as offering
different virtues.

On the practical front, psychologists are often interested in testing
how beliefs change given different forms of evidence. Most
investigations focus on testing for the existence of belief effects:
does a piece of evidence produce a change in peoples' beliefs? But other
investigations seek to uncover more nuanced patterns of belief updating.
In particular, many insights stand to be gained in examinations of
\emph{differential belief updating}, cases where certain individuals or
groups respond to the same evidence in different ways. For example: Do
Democrats and Republicans respond to scientific arguments in the same
way? Are emotional appeals persuasive, and for whom? Do people more
readily update their beliefs for good news than for bad?

Examinations of differential responding will generally need to quantify
degrees of belief change in a way that commits to a particular measure
that quantifies that degree of confirmation.\footnote{One exception
  could be extreme cases like ``backfire,'' where evidence has the
  opposite of its intended effect for some subset of people. Such cases
  are likely quite rare (Wood \& Porter, 2019), though they are at least
  theoretically possible (Jern et al., 2014).} As mentioned, the
theoretical literature has proposed many such measures, each with their
own virtues. Though rightly constrained by epistemic principles, a
psychological theory of confirmation has its own priorities and
desiderata that might help to narrow the field.

In this paper, we consider some of the most important confirmation
measures in the theoretical literature and which of them might best
fulfill the desiderata for a psychological theory of evidential
confirmation. We then present a study examining how competing Bayesian
confirmation measures track commonsense notions of evidential strength.
We demonstrate how these measures can be computed from participants'
belief reports, and identify cases where the measures come apart in
their characterization of participants' belief updating.

\hypertarget{confirmation-measures}{%
\subsection{Confirmation measures}\label{confirmation-measures}}

Theorists have proposed dozens of different incremental confirmation
measures. Some popular measures are the following:

\begin{itemize}
\item
  Difference Measure (Carnap, 1962; Earman, 1992):
  \[d(H,E)=P(H \mid E)-P(H)\]
\item
  Log Ratio Measure (Milne, 1996):
  \[r(H,E)=\log\left(\frac{P(H \mid E)}{P(H)}\right)\]
\item
  Log likelihood Measure (Fitelson, 1999; Good, 1983):
  \[l(H,E)=\log\left(\frac{P(E \mid H)}{P(E \mid \neg H)}\right)\]
\item
  Z-measure (Crupi, Tentori, \& Gonzalez, 2007): \[
  z(H,E) = \begin{cases}
  \frac{P(H \mid E) - P(H)}{1-P(H)}  &  P(H \mid E) \geq P(H) \\
  \frac{P(H \mid E) - P(H)}{P(H)} & P(H \mid E) < P(H)
  \end{cases}
  \]
\item
  Normalized Difference Measure (Christensen, 1999; Joyce, 1999):
  \[s(H,E)=P(H \mid E)-P(H \mid \neg E)\]
\end{itemize}

These measures all have attractive features and agree in many cases.
Still, there are important (ordinal) differences between them, and each
has been subject to serious criticism. Here, we highlight three main
areas of theoretical concern.

To begin, Eells and Fitelson (2002) use symmetries related to
confirmation to evaluate measures. Taking \(\mathbf{c}\) to be an
arbitrary confirmation measure, they consider the following symmetry
conditions:

\begin{itemize}
\item
  Evidence Symmetry (ES): \(\mathbf{c}(H,E)=-\mathbf{c}(H, \neg E)\)
\item
  Communitivity Symmetry (CS): \(\mathbf{c}(H,E)=-\mathbf{c}(E, H)\)
\item
  Hypothesis Symmetry (HS): \(\mathbf{c}(H,E)= \mathbf{c}(H, \neg E)\)
\end{itemize}

Eells and Fitelson argue ES and CS are not generally valid or desirable
while HS is. To see this, they consider an example in which a card is
drawn from a fair deck. Let \(H\) be the claim that the card is black
and \(E\) be the claim that the card is a seven of spades. That the card
is a seven of spades (\(E\)) is decisive evidence that the card is black
(\(H\)). But that the card is not a seven of spades (\(\neg E\)) is not
decisive evidence against the claim that the card is black (\(H\)). This
falsifies ES, which claims that the degree to which \(E\) confirms \(H\)
is equal to the degree to which \(\neg E\) disconfirms.

Additionally, that the card is black (\(H\)) is relatively weak evidence
that it is a seven of spades (\(E\)). This falsifies CS which claim the
degree to which \(E\) confirms \(H\) is the degree to which \(H\)
confirms \(E\).

On the other hand, the card being a seven of spades (\(E\)) maximally
disconfirms that claim that the card is not black (\(\neg H\)). So this
suggests HS is true. Indeed, other authors (Tentori, Crupi, Bonini, \&
Osherson, 2007) concur with Eells and Fitelson that HS is not subject to
counterexample.

Based on this, Eells and Fitelson conclude the correct confirmation
measure will validate HS, but not ES or CS. They show that \(d\) and
\(l\) do this. But \(r\) fails to validate HS and incorrectly validates
CS. And \(s\) incorrectly validates ES and CS and fails to validate HS.
More recently, (Crupi et al., 2007) have shown \(z\) correctly validates
HS, but not ES or CS.

Crupi, Tentori, and Gonzales (2007) have argued that there is a more
general family of desirable symmetry conditions. This larger family
includes the condition HS as one of its members, but it also includes
some more subtle conditions such as:
\[\text{if } P(H \mid E) <P(H)\text{, } \mathbf{c}(H,E)=\mathbf{c}(E,H)\]
The argument in favor of this more general family of symmetries is
sufficiently complex so as to be beyond the scope of this paper.
Nevertheless, what is important is that the authors show that only \(z\)
satisfies these more general symmetries.

Next, Tentori, Crupi, Bonini, and Oshershon (2007) present problems for
\(d\) and \(l\) related to whether they are natural generalizations of
logical entailment. If \(E\) entails \(H\), then plausibly confirmation
for \(H\) by \(E\) is maximal. But this does not hold for \(d\). To see
this, let \(H_1\) be the claim that the card is a seven, \(H_2\) be the
claim that the card is a spade, and \(E\) be the claim that the card is
a seven of spades. Here \(E\) provides maximal support for both \(H_1\)
and \(H_2\), but \(d(H_1,E) > d(H_2, E)\) due to the different prior
expectations \(P(H_1)\) and \(P(H_2)\).

\(l\) also faces trouble with generalizing logical entailment. This is
because it is undefined in cases where \(E\) entails \(H\) or entails
\(\neg H\) (because it involves division by \(0\) in the first case and
\(log(0)\) in the second case).

Finally, Fitelson (2001) observes that it is desirable for confirmation
measures to have an additive feature in cases where pieces of evidence
are independent. To state Fitelson's claim, it helps to introduce some
terminology. Given a probability distribution \(P\), let \(P_E\) be
defined as \(P_E(\cdot)=P(\cdot \mid E)\). And given a confirmation
measure \(\mathbf{c}\) defined by \(P\), let \(\mathbf{c}_E\) be exactly
like \(\mathbf{c}\) except defined by \(P_E\). Finally let us say:

\begin{quote}
\(E_1\) and \(E_2\) are \emph{confirmationally independent regarding}
\(H\) \emph{according to} \(\mathbf{c}\) exactly if
\(\mathbf{c}(H, E_1)=\mathbf{c}_{E_2}(H, E_1)\) and
\(\mathbf{c}(H, E_2)=\mathbf{c}_{E_1}(H, E_2)\)
\end{quote}

When \(E_1\) and \(E_2\) are confirmationally independent regarding
\(H\), the confirmation measure says that updating on one piece of
evidence makes no difference to the degree of confirmation provided to
\(H\) by the other piece of evidence.

Fitelson claims that the following is desideratum for a confirmation
measure:

\begin{quote}
If \(E_1\) and \(E_2\) are confirmationally independent regarding \(H\)
according to \(\mathbf{c}\), then
\(\mathbf{c}(H, E_1 \wedge E_2)=\mathbf{c}(H, E_1) + \mathbf{c}(H, E_2)\)
\end{quote}

It can be shown that \(d\), \(r\), and \(l\) satisfy this condition. But
it can be shown that nothing like this holds for \(s\).

\(z\) also does not satisfy this condition. However, it can be shown for
\(z\), that the relevant conditions holds wherever \(E_1\) and \(E_2\)
both individually confirm \(H\) or \(E_1\) and \(E_2\) both individually
disconfirm \(H\). But in cases where \(E_1\) and \(E_2\) ``point in
different directions'', the additivity condition does not hold
(Fitelson, 2021).

Overall, criteria related to symmetries, generalizing of entailment, and
additivity of independent evidence provide support for certain measures
and count against others. We believe that \(l\) and \(z\) fare best
based on these principled considerations, though the arguments are not
entirely decisive.

\hypertarget{desiderata-for-a-psychological-theory-of-confirmation}{%
\subsection{Desiderata for a psychological theory of
confirmation}\label{desiderata-for-a-psychological-theory-of-confirmation}}

The foregoing principled considerations define desiderata for an
epistemological confirmation measure. What then for a psychological
theory of confirmation? A psychological theory ought to be based upon
these epistemological principles, but has its own considerations that
will lead them to be weighted differently. We will argue that \(l\)
provides the most satisfying criteria for a psychological theory of
confirmation.

First, psychological studies using self reports and examining
naturalistic sources of evidence (i.e.~beyond the realm of balls and
urns) can derive measures for prior and posterior beliefs for some
hypothesis \(H\), but it is typically not obvious how to identify other
evidential or intermediate belief values, such as \(P(H|\neg E)\). Thus,
it is more or less a prerequisite that the measures be estimable only
from these measures (obviating measures like \(s\)). \(d\), \(r\), and
\(z\) are quite obvious to estimate from their definitions. \(l\) can
also be directly estimated from prior and posterior belief reports,
leveraging the log-odds form of Bayes' Rule:

\[\log O(H|E) = \log O(H) + \log\left(\frac{P(E \mid H)}{P(E \mid \neg H)}\right)\]

So that \(l\) can be estimated as:

\[l(H,E) = \log O(H|E) - \log O(H)\]

Second, certain principled desiderata seem less important for a
psychological theory. One criticism against \(l\) (and \(d\)) is that it
does not naturally generalize logical entailment. However, this concern
is of substantially less importance for psychological theory, where we
might more reasonably hypothesize there is a discontinuity between
inductive and deductive modes of reasoning.\footnote{It is also worth
  noting that the failures of \(l\) to generalize logical entailment can
  be quite naturally alleviated by defining \(l\) in terms of its limits
  of \(+ \infty\) or \(- \infty\) wherever it would otherwise be
  undefined.

  \{redacted for blind review\} suggests to us a more principled way of
  achieving the same result that is based on a correction of an
  observation made by I.J. Good (1975). It is known that the following
  measure is ordinally equivalent to
  \(l\):\[K(H,E)=\frac{Pr(E \mid H)-Pr(E \mid \neg H)}{Pr(E \mid H)+Pr(E \mid \neg H)}\]
  Further, it is known that putting the extreme cases
  aside:\[l(H,E)=2 \times \text{ArcTanh}\left(K(H,E)\right)\]But since
  according to the standard definition \(\text{ArcTanh}(1)=+\infty\) and
  \(\text{ArcTanh}(-1)=-\infty\). We may take the above to be a fully
  general new definition of \(l\) that is defined in the extreme cases.}

Finally, a psychological theory of confirmation should serve a
meaningful role in the larger program of Bayesian cognitive science and
psychology. Bayesian cognitive scientists have argued that much of
higher-order reasoning is subserved by generative probabilistic mental
models (Battaglia, Hamrick, \& Tenenbaum, 2013; Chater et al., 2020;
Tenenbaum, Kemp, Griffiths, \& Goodman, 2011). These mental models
represent people's understanding of a domain, allowing them to make
inferences and predictions and to reason about new evidence. For
psychologists working in this paradigm, describing these mental models
is a chief concern and examining differential belief updating is one
potentially powerful lens through which they might be better understood.
In some cases we might be interested in whether certain individuals
update their beliefs more or less rationally, e.g.~by examining whether
motivational factors influence the degree to which people revise their
beliefs (e.g. Hahn \& Harris, 2014; Möbius, Niederle, Niehaus, \&
Rosenblat, 2022; Powell, 2022; Sharot, Korn, \& Dolan, 2011). In other
cases, we might use differential belief updating to test whether people
have different intuitive theories or auxiliary beliefs regarding the
evidence (Gershman, 2019; Jern, Chang, \& Kemp, 2014). Identifying such
cases could offer important clues to the larger mental models people use
to reason about evidence in important domains such as vaccination
decisions (e.g. Powell, Weisman, \& Markman, 2023), climate change (e.g.
Cook \& Lewandowsky, 2016; Schotsch \& Powell, 2022), and other major
issues.

First, evaluations of the rationality of human belief updating must be
made with respect to some normative standard. Typically, these concern
simple cases where the observed evidence has a known impact, described
by Bayes' Rule in terms of the likelihood of the data given the
hypotheses in contention (e.g. Edwards, 1968). A question is whether
people update their beliefs according to this likelihood. Here,
comparisons according to \(l\) will typically offer the most direct
answer, as this measure is derived entirely from the likelihoods.

Second, psychologists investigating people's mental models of a domain,
or otherwise seeking to compare human behavior against a rational
Bayesian standard, ought to be especially interested in examining when
mental models or auxiliary beliefs specify different theories of the
evidence (Gershman, 2019).

Consider Alice and Bob, a doctor and patient awaiting the results of a
diagnostic test for an uncommon medical condition (\(H\)). The patient,
Bob, is quite nervous, and holds a prior belief that he has the
condition \(P_{B}(H) = .50\). In contrast, Alice knows that the
condition is really quite rare and so holds a much more skeptical prior,
\(P_{A}(H) = .10\). However, both doctor and patient agree perfectly in
their understanding of the diagnostic test, understanding it to have
specificity and sensitivity of .90, i.e.~\(P(E|H) = .90\) and
\(P(E| \neg H) =. 10\). Fortunately for Bob, the test returns negative.
Both Alice and Bob revise their beliefs rationally according to Bayes'
Rule. Bob's posterior belief he has the condition is now .10 and Alice's
is approximately .01.

Psychologically, we might ask whether Alice and Bob responded to the
evidence in different ways? That is, we might ask for a comparison of
their confirmation under some measure, confirmation simply being the
degree to which a belief was updated by evidence. Within the Bayesian
program however, this psychological question might also be seen as
asking whether Alice and Bob differ in their mental models of the
evidence (\(\mathcal{M}\)), i.e.~whether their likelihood distribution
\(P_{\mathcal{M}_a}(E|H) = P_{\mathcal{M}_b}(E|H)\) (where here we use
\(E\) and \(H\) to indicate R.V.). Comparisons of \(l\) will most
directly answer this psychological question.

\begin{CodeChunk}
\begin{figure*}[!ht]
\includegraphics{figs/fig1-1} \caption[Beliefs before and after exposure to evidence across topics for all participants]{Beliefs before and after exposure to evidence across topics for all participants. Each participants' pretest and posttest belief reports are represented with two points connected by a line.}\label{fig:fig1}
\end{figure*}
\end{CodeChunk}

\hypertarget{the-present-study}{%
\subsection{The present study}\label{the-present-study}}

To demonstrate how different confirmation measures can come apart and
therefore the need to commit to such a measure to make meaningful claims
about differential belief updating, we examined how these measures map
on to commonsense notions of evidential strength. Just as people readily
deploy commonsense or folk-psychological notions of psychological
concepts like ``belief'' or ``desire'', they are likewise perfectly
willing to discuss the persuasiveness of different sources of evidence.
By measuring how people's beliefs changed in light of several pieces of
evidence, and how those individuals rated the persuasiveness of that
evidence, we sought to identify the measure of confirmation to which
these ordinary language ascriptions best correspond.

To our knowledge, research from Crupi, Tentori and colleagues (Crupi et
al., 2007; Tentori et al., 2007) has produced the only empirical data
addressing this type of question previously. Using an abstract ``urns''
task, they compared participant's probability judgments and their
ratings of evidential impact following their observations of ball draws.
Translating participants' probability judgments into various measures of
confirmation. Analyses of their data found that \(l\) and \(z\) (Crupi
et al., 2007) were the measures most strongly correlated with
participants' impact judgments, with some evidence that \(z\) provided
the strongest correlation.

Our study explores this question in a naturalistic rather than
artificial context, exploring these relationships among beliefs and
evidence related to consequential real-world domains.\footnote{All
  materials and analysis code are available at \{github link redacted
  for blind review\}.}

\hypertarget{methods}{%
\section{Methods}\label{methods}}

\hypertarget{participants}{%
\subsection{Participants}\label{participants}}

A total of 217 participants were recruited from Amazon Mechanical Turk
(mTurk) through CloudResearch. These participants were all from the U.S.
and were at least 18 years of age. Participants who failed a basic
attention check question (17) were excluded from analysis. The final
sample in the analyses reported below was 200 (80 female, 118 male,
median age 36 years).

\hypertarget{materials-and-procedures}{%
\subsection{Materials and procedures}\label{materials-and-procedures}}

Four brief educational vignettes were created to correct common
misconceptions about four different topics. The four topics were 1) the
anthropogenic nature of climate change, 2) the dangers of skin contact
with Fentanyl, 3) the effectiveness of education tailored to individual
``learning styles,'' and 4) the economic impacts of major sports stadium
construction.

Each of these vignettes were designed to provide evidence for or against
a more specific belief. Participants were asked to report their beliefs
in terms of probabilities. For instance, evidence about the economic
impacts of sports stadiums was paired with a question asking
participants to judge the probability that a new sports stadium being
built in Buffalo would generate enough tax revenue to pay a return on
the public investment.

Participants were asked about the topics in a randomized order. For each
topic, participants made an initial probability judgment (pretest), then
read the brief educational intervention, and then were asked to make a
second probability judgment in light of what they had read (posttest).
Then, in a second phase of the study, participants were asked to rate
how persuasive they had found the evidence to be. These ratings were
made on a Likert scale from ``Not at all persuasive'' to ``Extremely
persuasive''.

\begin{CodeChunk}
\begin{figure*}[htb]

{\centering \includegraphics{figs/fig2-1} 

}

\caption[Average confirmation across persuasiveness ratings for each confirmation measure (facets) and topic (line colors)]{Average confirmation across persuasiveness ratings for each confirmation measure (facets) and topic (line colors). Error bars represent one standard error. Persuasiveness ratings concern confirmatory evidence for the climate topic and disconfirmatory evidence for all others. Thus, a well-behaved measure would be indicated by a monotonically increasing trend for climate and monotonically decreasing trends for all other topics.}\label{fig:fig2}
\end{figure*}
\end{CodeChunk}

\hypertarget{results}{%
\section{Results}\label{results}}

Figure 1 shows participants' probability judgments before and after
reading information about each of the four topics. As intended, each
piece of evidence had a substantial impact on beliefs, though the
magnitude of this effect appears to vary across topics.

Several different measures can be used to quantify the degree to which
participants' beliefs were affected. We focus on four indices that have
been advocated for in the literature on Bayesian Confirmation measures
and that can be computed from measures of belief before and after
observation of evidence: Probability differences (\(d\)), log
probability ratios (\(r\)), log likelihood ratios (\(l\)), and the
Z-measure (\(z\)).

For each participant, each of these belief updating measures was
computed for each topic. Figure 2 shows the average of these measures
against participants' ratings of the persuasiveness of the evidence. By
comparing different measures of participant's belief updating against
their persuasiveness ratings, we can probe which confirmation measure
best corresponds with common-sense notions of evidential strength.

Given that our persuasiveness scale provides a face-valid ordinal
measure of persuasiveness, we should expect to observe a monotonic
association between confirmation and persuasiveness ratings.
Particularly damning for such a correspondence would be
\emph{population-level reversals}, cases where average confirmation
values reliably reversed their ordering across levels of the ordinal
scale. As shown in Figure 2, this clearly occurs for \(d\), \(r\), and
\(z\) measures in the case of the evidence about climate change.
Wilcoxon rank sum tests comparing measures for confirmation ratings at
the midpoint (3) versus the high endpoint (5) reveal these differences
are reliable (all \(Ps < .05\))---in each case the degree of
confirmation among participants rating the evidence ``Extremely
Persuasive'' was significantly weaker than for those rating it
``Moderately Persuasive''. In contrast, \(l\) shows no reliable
reversals for any of the topics (all \(Ps > .20\)). Although values of
\(l\) do appear to plateau across persuasiveness levels for the climate
change topic, monotonicty is not violated.

It is likely that the reversals observed for \(d\), \(r\), and \(z\) in
the case of climate change beliefs owe to the strong correlation between
persuasiveness ratings and prior beliefs (Figure 3). \(l\) is unaffected
by this correlation as it is independent of prior beliefs \(p(H|K)\)
(though it may be somewhat affected by rounding or other measurement
error, especially near the bounds of the probability scale). Since \(z\)
is scaled by the prior we might have expected it to be been less
affected by this correlation than \(d\), yet it was nevertheless
observed to exhibit reversals.

\begin{CodeChunk}
\begin{figure}[tb]

{\centering \includegraphics{figs/fig3-1} 

}

\caption[Scatterplots showing correlations between prior beliefs and persuasiveness ratings]{Scatterplots showing correlations between prior beliefs and persuasiveness ratings. To improve readability, points have been jittered along their persuasiveness ratings.}\label{fig:fig3}
\end{figure}
\end{CodeChunk}

\hypertarget{discussion}{%
\section{Discussion}\label{discussion}}

Of the four Bayesian confirmation measures we examined, we found that
\(l\) best tracked common sense notions of evidential strength. We
assessed the viability of each of these measures by examining how they
correspond with ratings of the persuasiveness of evidence. All three
other measures, \(d\), \(r\), and \(z\) failed to consistently track
persuasive ratings in a monotonic fashion. Instead, each of these
measures exhibited at least one population-level reversal, where higher
persuasiveness ratings were associated with reliably lower values on the
confirmation measure.

Prior findings by Crupi, Tentori, and colleagues found \(l\) and \(z\)
to best track human judgments of evidential impact (Crupi et al., 2007;
Tentori et al., 2007) . These researchers examined the correlation
between impact judgments and belief updates across many instances of
evidence in a simple ball-and-urn task, finding that \(l\) and \(z\)
better correlated than did other measures like \(d\) and \(r\). Our work
both expands upon and refine prior findings. First, our findings extend
the psychological investigation of confirmation measures to more
naturalistic contexts, concerning the sort of evidence that might
influence real-world decisions. Second, our findings provide empirical
support for \(l\) exclusively, while more sharply demonstrating the
shortcomings of \(d\), \(r\), and \(z\). Our findings reveal clear
failures of these measures, showing that they violate basic measurement
constraints on the relationship between persuasiveness judgments and
belief updating.

How do these empirical findings relate back to the two theoretical
projects we considered at the outset, epistemological and psychological
theories of confirmation? Although these theoretical concerns motivate
our interest in this empirical psychological question, we see this
empirical project as largely descriptive. We saw that \(l\) fares well
according to a variety of principled epistemological criteria and that
there are often independent reasons for psychologists to prefer \(l\) as
a measure of confirmation. Our finding that \(l\) also comports with
common sense notion does not address any outstanding theoretical
criticisms, though we do see it as another virtue in its favor.

Finally, we consider some potential limitations and directions for
future explorations. First, there are some issues related to our measure
of persuasiveness that may warrant further investigation. We measured
participants' assessments of evidential strength by asking them to rate
how ``persuasive'' the evidence was. However, there could be some
concerns about how participants answer this question---whether they do
so as immediately (i.e., how persuasive was it to them in this
instance?) or broadly (e.g.~how persuasive would it be to some one
else?) or counterfactually (e.g.~if you were just hearing this for the
first time, how persuasive would it be?).

These interpretations may raise questions related to the philosophical
problem of old evidence. The problem arises in cases where one already
regards \(P(E)\) to be close to \(1\) but nonetheless sees \(E\) as
important evidence for \(H\). It may be that some subjects already know
about the evidence provided by the vignette, so they can be thought of
as having \(P(E)\) close to \(1\)---though \(E\) might not shift their
beliefs in \(H\) in this moment, they might still regard it as highly
persuasive. To be sure, cases like this are not unique to our
experimental context: Glymour (1980) has argued that there are a number
of prominent cases like this in the history of science. Many
confirmation measures struggle to adequately deal with these kinds of
examples (Christensen, 1999; Glymour, 1980), though it has been argued
that \(l\) fares well with certain aspects of this problems (Eells \&
Fitelson, 2000).

Second, there are some potentially interesting cases our empirical study
has not examined. One set of cases would be strong evidence for highly
implausible or against nearly certain claims. For instance, the kind of
evidence that could move someone from \(P(H) = .01\) to \(P(H|E) = .1\).
Such a case of evidence would score a relatively large value for \(l\)
(approximately equivalent to moving from \(P(H) = .50\) to
\(P(H|E) = .90\)), but would still leave a reasoner quite skeptical. It
is not intuitive to imagine how people would rate such evidence, and
these sorts of cases may pose problems for intuitions about \(l\) not
identified here.

Finally, there are persistent biases in human probability judgments that
pose a general challenge to measurement in this arena (Kahneman, 2011;
e.g. Kahneman, Slovic, \& Tversky, 1982). Two recent theories of
probability judgments have explained a host of observed biases by
proposing that people's probability judgments are shrunk toward .50 by
varying degrees (Costello \& Watts, 2014; Zhu, Sanborn, \& Chater,
2020). This scaling could potentially induce non-monotonicity in
observed updates despite equivalent ``true'' updates for the measures
\(l\), \(r\), and \(z\), so it is possible the shortcomings of \(r\) and
\(z\) may be partially explained by these biases. Nevertheless, as
calculated from observed probability judgments, \(l\) provides a measure
of confirmation that reliably comports with commonsense notions.

\hypertarget{conclusions}{%
\subsection{Conclusions}\label{conclusions}}

Rigorous investigations into \emph{differential belief updating} demand
a psychological theory of confirmation. Drawing on measures of
confirmation identified in the epistemological literature, we identified
\(l\) as the measure that most directly addresses the concerns of
psychologists investigating Bayesian and non-Bayesian belief updating.
In an experimental study, we demonstrated how this and other measures
can be computed from participants' belief reports, and identify cases
where the measures come apart in tracking participants' independent
assessments of evidence Our findings indicate that, in addition to its
theoretical virtues, \(l\) is also the measure that best characterizes
commonsense notions of belief updating.

\hypertarget{references}{%
\section{References}\label{references}}

\setlength{\parindent}{-0.1in} 
\setlength{\leftskip}{0.125in}

\noindent

\hypertarget{refs}{}
\begin{CSLReferences}{1}{0}
\leavevmode\vadjust pre{\hypertarget{ref-battaglia.etal2013}{}}%
Battaglia, P. W., Hamrick, J. B., \& Tenenbaum, J. B. (2013). Simulation
as an engine of physical scene understanding. \emph{Proceedings of the
National Academy of Sciences}, \emph{110}(45), 18327--18332.
http://doi.org/\href{https://doi.org/10.1073/pnas.1306572110}{10.1073/pnas.1306572110}

\leavevmode\vadjust pre{\hypertarget{ref-Carnap1962}{}}%
Carnap, R. (1962). \emph{Logical foundations of probability} (2nd ed.).
Chicago: The University of Chicago Press.

\leavevmode\vadjust pre{\hypertarget{ref-chater.etal2020}{}}%
Chater, N., Zhu, J.-Q., Spicer, J., Sundh, J., León-Villagrá, P., \&
Sanborn, A. (2020). Probabilistic {Biases} {Meet} the {Bayesian}
{Brain}. \emph{Current Directions in Psychological Science},
\emph{29}(5), 506--512.
http://doi.org/\href{https://doi.org/10.1177/0963721420954801}{10.1177/0963721420954801}

\leavevmode\vadjust pre{\hypertarget{ref-Christensen1999}{}}%
Christensen, D. (1999). Measuring confirmation. \emph{Journal of
Philosophy}, \emph{96}, 437--461.

\leavevmode\vadjust pre{\hypertarget{ref-cook.lewandowsky2016}{}}%
Cook, J., \& Lewandowsky, S. (2016). Rational {Irrationality}:
{Modeling} {Climate} {Change} {Belief} {Polarization} {Using} {Bayesian}
{Networks}. \emph{Topics in Cognitive Science}, \emph{8}(1), 160--179.
http://doi.org/\href{https://doi.org/10.1111/tops.12186}{10.1111/tops.12186}

\leavevmode\vadjust pre{\hypertarget{ref-costello.watts2014}{}}%
Costello, F., \& Watts, P. (2014). Surprisingly rational: {Probability}
theory plus noise explains biases in judgment. \emph{Psychological
Review}, \emph{121}(3), 463--480.
http://doi.org/\href{https://doi.org/10.1037/a0037010}{10.1037/a0037010}

\leavevmode\vadjust pre{\hypertarget{ref-CTG2007}{}}%
Crupi, V., Tentori, K., \& Gonzalez, M. (2007). On bayesian measures of
evidential support. \emph{Philosophy of Science}, \emph{74}, 229--252.

\leavevmode\vadjust pre{\hypertarget{ref-Earman1992}{}}%
Earman, J. (1992). \emph{Bayes or bust}. Cambridge: MIT Press.

\leavevmode\vadjust pre{\hypertarget{ref-edwards1968}{}}%
Edwards, W. (1968). Conservatism in {Human} {Information} {Processing}.
In B. Kleinmuntz (Ed.), \emph{Formal representation of human judgment}
(pp. 17--52). New York: Wiley.

\leavevmode\vadjust pre{\hypertarget{ref-EF2000}{}}%
Eells, E., \& Fitelson, B. (2000). Measuring confirmation and evidence.
\emph{The Journal of Philosophy}, \emph{97}, 663--672.

\leavevmode\vadjust pre{\hypertarget{ref-EF2002}{}}%
Eells, E., \& Fitelson, B. (2002). Symmetries and asymmetries in
evidential support. \emph{Philosophical Studies}, \emph{107}, 129--142.

\leavevmode\vadjust pre{\hypertarget{ref-Fitelson1999}{}}%
Fitelson, B. (1999). The plurality of bayesian measures of confirmation
and the problem of measure sensitivity. \emph{Philosophy of Science},
\emph{66}, S362--S378.

\leavevmode\vadjust pre{\hypertarget{ref-Fitelson2001}{}}%
Fitelson, B. (2001). A bayesian account of independent evidence with
applications. \emph{Philosophy of Science}, \emph{68}, S123--S140.

\leavevmode\vadjust pre{\hypertarget{ref-Fitelson2021}{}}%
Fitelson, B. (2021). A problem for confirmation measure z.
\emph{Philosophy of Science}, \emph{88}, 726--730.

\leavevmode\vadjust pre{\hypertarget{ref-gershman2019}{}}%
Gershman, S. J. (2019). How to never be wrong. \emph{Psychonomic
Bulletin \& Review}, \emph{26}(1), 13--28.
http://doi.org/\href{https://doi.org/10.3758/s13423-018-1488-8}{10.3758/s13423-018-1488-8}

\leavevmode\vadjust pre{\hypertarget{ref-Glymour1980}{}}%
Glymour, C. (1980). \emph{Theory and evidence}. Princeton: Princeton
University Press.

\leavevmode\vadjust pre{\hypertarget{ref-Good1975}{}}%
Good, I. J. (1975). Explicativity, corroboration, and the relative odds
of hypotheses. \emph{Synthese}, \emph{30}, 39--73.

\leavevmode\vadjust pre{\hypertarget{ref-Good1983}{}}%
Good, I. J. (1983). \emph{Good thinking}. Minneapolis: University of
Minnesota Press.

\leavevmode\vadjust pre{\hypertarget{ref-hahn.harris2014}{}}%
Hahn, U., \& Harris, A. J. L. (2014). What {Does} {It} {Mean} to be
{Biased}. In \emph{Psychology of {Learning} and {Motivation}} (Vol. 61,
pp. 41--102). Elsevier.
http://doi.org/\href{https://doi.org/10.1016/B978-0-12-800283-4.00002-2}{10.1016/B978-0-12-800283-4.00002-2}

\leavevmode\vadjust pre{\hypertarget{ref-jern.etal2014}{}}%
Jern, A., Chang, K. K., \& Kemp, C. (2014). Belief polarization is not
always irrational. \emph{Psychological Review}, \emph{121}(2), 206--224.
http://doi.org/\href{https://doi.org/10.1037/a0035941}{10.1037/a0035941}

\leavevmode\vadjust pre{\hypertarget{ref-Joyce1999}{}}%
Joyce, J. (1999). \emph{The foundations of causal decision theory}.
Cambridge: Cambridge University Press.

\leavevmode\vadjust pre{\hypertarget{ref-kahneman2011}{}}%
Kahneman, D. (2011). \emph{Thinking, {Fast} and {Slow}} (1st edition).
New York: Farrar, Straus; Giroux.

\leavevmode\vadjust pre{\hypertarget{ref-kahneman.etal1982}{}}%
Kahneman, D., Slovic, P., \& Tversky, A. (1982). \emph{Judgment {Under}
{Uncertainty}: {Heuristics} and {Biases}}. (D. Kahneman, P. Slovic, \&
A. Tversky, Eds.). Cambridge University Press.

\leavevmode\vadjust pre{\hypertarget{ref-Milne1996}{}}%
Milne, P. (1996). \emph{Log{[}p(h/eb)/p(h/b){]}} is the one true measure
of confirmation. \emph{Philosophy of Science}, \emph{63}, 21--26.

\leavevmode\vadjust pre{\hypertarget{ref-mobius.etal2022}{}}%
Möbius, M. M., Niederle, M., Niehaus, P., \& Rosenblat, T. S. (2022).
Managing {Self}-{Confidence}: {Theory} and {Experimental} {Evidence}.
\emph{Management Science}, \emph{68}(11), 7793--7817.
http://doi.org/\href{https://doi.org/10.1287/mnsc.2021.4294}{10.1287/mnsc.2021.4294}

\leavevmode\vadjust pre{\hypertarget{ref-powell2022}{}}%
Powell, D. (2022). A descriptive bayesian account of optimism in belief
revision. In C. Jennifer, A. Perfors, H. Rabagliati, \& V. Ramenzoni
(Eds.), \emph{Proceedings of the 42nd {Annual} {Conference} of the
{Cognitive} {Science} {Society}}.

\leavevmode\vadjust pre{\hypertarget{ref-powell.etal2023}{}}%
Powell, D., Weisman, K., \& Markman, E. M. (2023). Modeling and
leveraging intuitive theories to improve vaccine attitudes.
\emph{Journal of Experimental Psychology: General}.

\leavevmode\vadjust pre{\hypertarget{ref-schotsch.powell2022}{}}%
Schotsch, B., \& Powell, D. (2022). Understanding intuitive theories of
climate change. In J. Culbertson, A. Perfors, H. Rabagliati, \& V.
Ramenzoni (Eds.), \emph{Proceedings of the 44th {Annual} {Meeting} of
the {Cognitive} {Science} {Society}}. Austin, TX: Cognitive Science
Society.

\leavevmode\vadjust pre{\hypertarget{ref-sharot.etal2011}{}}%
Sharot, T., Korn, C. W., \& Dolan, R. J. (2011). How unrealistic
optimism is maintained in the face of reality. \emph{Nature
Neuroscience}, \emph{14}(11), 1475--1479.
http://doi.org/\href{https://doi.org/10.1038/nn.2949}{10.1038/nn.2949}

\leavevmode\vadjust pre{\hypertarget{ref-tenenbaum.etal2011}{}}%
Tenenbaum, J. B., Kemp, C., Griffiths, T. L., \& Goodman, N. D. (2011).
How to {Grow} a {Mind}: {Statistics}, {Structure}, and {Abstraction}.
\emph{Science}, \emph{331}(6022), 1279--1285.
http://doi.org/\href{https://doi.org/10.1126/science.1192788}{10.1126/science.1192788}

\leavevmode\vadjust pre{\hypertarget{ref-TCBO2007}{}}%
Tentori, K., Crupi, V., Bonini, N., \& Osherson, D. (2007). Comparision
of confirmation measures. \emph{Cognition}, \emph{103}, 107--119.

\leavevmode\vadjust pre{\hypertarget{ref-zhu.etal2020}{}}%
Zhu, J.-Q., Sanborn, A. N., \& Chater, N. (2020). The {Bayesian}
sampler: {Generic} {Bayesian} inference causes incoherence in human
probability judgments. \emph{Psychological Review}, \emph{127}(5),
719--748.
http://doi.org/\href{https://doi.org/10.1037/rev0000190}{10.1037/rev0000190}

\end{CSLReferences}

\bibliographystyle{apacite}


\end{document}
